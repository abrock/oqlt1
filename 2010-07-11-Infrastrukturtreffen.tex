\documentclass{scrartcl}

\usepackage[utf8]{inputenc}
\usepackage{microtype}

\newcommand\docsteel{Docsteel}
\newcommand\euro{Euro}
\newcommand\uc{Mikrocontroller}

\begin{document}

\tableofcontents

\section{Einleitung}

Tim Weber begrüßt die Anwesenden und erklärt, warum er die Projektgruppe Infrastruktur ins Leben gerufen hat.

\section{Netzwerkstruktur}

Alle Netzwerkkabel auf das Patchpanel legen, spezielle Dienste mit merkwürdigen EIgenschaften (Stromzähler...) gesondert beschriften
um Verwirrung zu vermeiden.

Tim betont den Aspekt der (kurz- und Langfristigen) Wirtschaftlichkeit.

Niklas will das Rack zum Lagern von Netzwerkhardware verwenden.

Tim will die Firebox nicht für allen möglichen Firlefanz einsetzen.
Deshalb soll ein zweiter Server für interne Dienste angeschafft werden.

Der Internet-Uplink geht durch den Server von \docsteel.

Wir brauchen einen Notfallplan für Dummies, den wir an das Rack hängen und der erklärt, was man tut, wenn
das Internet ausfällt.

Der Switch von Timo soll sobald wie möglich durch etwas besseres ersetzt werden.
Der 3Com Superstack fällt wegen bekannter Merkwürdigkeiten aus.
Die letzten 6 Ports am Switch sind für interne Dienste und sollen entsprechen beschriftet werden.

Tim schlägt vor, den Switch abzuschalten, wenn niemand im Raum ist.
\docsteel will den WLAN-AP direkt an die Firebox hängen.
Der Flukso muss dann auch direkt an die Firebox.

Die Firebox hat 5 Ports +1 Uplink-Port. Die Ports werden wie folgt belegt:
\begin{itemize}
\item{} Switch
\item{} WLAN-AP
\item{} Management
\item{} Flukso
\item{} Mikrocontroller
\end{itemize}

Die Firebox muss dann irgendwie die IP-Steckdosen bedienen.

Wir brauchen schaltbare Steckdosen und einen Automatisumus, der alles unnötiges abschaltet, wenn niemand mehr da ist.

Auf https://datenschleuder.net/munin/net/datenschleuder.net-if\_dn42\_rzl.html stehen Traffic-Statistiken des Uplinks.

Für interne Dienste steht die Blackbox von Inte zur Verfügung.

\section{Zentrale Benutzerverwaltung}

Die zentrale Benutzerverwaltung soll die Getränkekasse und die Tür-Codes verwalten.

\section{Türschloss}

Tim hat ein ABUS Hometech für 60\euro bestellt.
Da der aktuelle Schließzylinder steht nicht weit genug über den Beschlag über, dieses Problem kann man auf zwei Arten lösen:
\begin{itemize}
\item{} Anderen Schließzylinder kaufen
\item{} Den Beschlag passen aussparen
\end{itemize}

Möglichkeiten, sich zu authentifizieren:
\begin{itemize}
\item{} Pinpad außen an der Tür
\item{} Tresorschloss mittels Drehencoder
\item{} SSH-Login per WLAN
\end{itemize}

Wir brauchen eine Möglichkeit, anzuzeigen, dass man den Raumstatus nicht auf `"auf'" setzen will.
Wir wollen die Schließvorgänge protokollieren, um bei Missbrauch nachvollziehen zu können, wer verantwortlich ist.
Wer möchte, kann denn Statusbot anzeigen lassen, dass er anwesend ist.

\section{Verbindung von Mikrocontrollern}

Möglichkeit: 6lowpan
Problem: Kosten 20\euro pro Modul

Anderer Ansatz: RS485, Kabel. Dazu wird ein Gateway-Controler benötigt, der als Router für das \uc-Netz fungiert.
Als Gateway könnte man ein NET-IO-Board von Pollin nehmen.

Tim hat für dieses System 10 Treibermodule für insgesamt 6\euro gekauft

\section{Kassenterminal}

Anforderungen:
\begin{itemize}
\item{} Barcode-Scanner
\item{} \uc
\item{} Punktmatrix-Display
\item{} Über den Bus IP sprechen und mit der Kassensoftware kommunizieren.
\item{} Audio-Feedback
\item{} Kopplung an den Statusbot für personalisierte Anzeige
\end{itemize}

\section{Lichtsteuerung}

Soll ebenfalls über den RS485-Bus laufen.
Tim redet mit DaFo wegen seinem NET-IO-Board und Lichtsteuerung.
Wir wollen Rauchmelder, die an \uc angeschlossen sind und uns alarmieren, wenn es brennt.
Wir wollen die Ätzecke komplett abschalten, wenn niemand da ist.


\section{Eigenes Internet}

\begin{itemize}
\item{} \docsteel hat jemanden bei Alstom angeschrieben, aber noch keine Rückmeldung erhalten.
\item{} Im Gebäude liegen Leerrohre/Glasfaser, die man dafür nutzen könnte.
\item{} Inte erklärt sich bereit, da nachzuhaken.
\item{} \docsteel fragt bei Claranet
\item{} Niklas fragt bei QSC
\end{itemize}

\section{Audioinfrastruktur}

\begin{itemize}
\item{} Wir wollen die 4 neuen Boxen an die Anlage anschließen, wissen aber nicht, ob die für das Audiolabor gedacht waren.
\item{} Wir wollen ein symmetrisches Signal vom Audiolabor zur Anlage. Dafür brauchen wir ein Gerät an der Anlage, das das kann.
\item{} 
\item{} 
\item{} 
\end{itemize}

\section{Zeug in Kühlschränken}

Problem: Essen wird in die Kühlschränke gelegt und vergessen.

Lösung: Essen wird mit Aufklebern mit Name und Datum beschriftet und nach \$Zeit vogelfrei.



Netzwerkstruktur
Streaming von Vorträgen

\end{document}
