\documentclass{scrartcl}

\usepackage[utf8]{inputenc}
\usepackage{microtype}

\newcommand\docsteel{Docsteel}



\begin{document}

\tableofcontents

\section{Einleitung}

Tim Weber begrüßt die Anwesenden und erklärt, warum er die Projektgruppe Infrastruktur ins Leben gerufen hat.

\section{Netzwerkstruktur}

Alle Netzwerkkabel auf das Patchpanel legen, spezielle Dienste mit merkwürdigen EIgenschaften (Stromzähler...) gesondert beschriften
um Verwirrung zu vermeiden.

Tim betont den Aspekt der (kurz- und Langfristigen) Wirtschaftlichkeit.

Niklas will das Rack zum Lagern von Netzwerkhardware verwenden.

Tim will die Firebox nicht für allen möglichen Firlefanz einsetzen.
Deshalb soll ein zweiter Server für interne Dienste angeschafft werden.

Der Internet-Uplink geht durch den Server von \docsteel.

Wir brauchen einen Notfallplan für Dummies, den wir an das Rack hängen und der erklärt, was man tut, wenn
das Internet ausfällt.

Der Switch von Timo soll sobald wie möglich durch etwas besseres ersetzt werden.
Der 3Com Superstack fällt wegen bekannter Merkwürdigkeiten aus.
Die letzten 6 Ports am Switch sind für interne Dienste und sollen entsprechen beschriftet werden.

Tim schlägt vor, den Switch abzuschalten, wenn niemand im Raum ist.
\docsteel will den WLAN-AP direkt an die Firebox hängen.
Der Flukso muss dann auch direkt an die Firebox.

Die Firebox hat 5 Ports +1 Uplink-Port. Die Ports werden wie folgt belegt:
\begin{itemize}
\item{} Switch
\item{} WLAN-AP
\item{} Management
\item{} Flukso
\item{} Mikrocontroller
\end{itemize}

Die Firebox muss dann irgendwie die \steckdosen bedienen.

Wir brauchen schaltbare Steckdosen und einen Automatisumus, der alles unnötiges abschaltet, wenn niemand mehr da ist.

Auf https://datenschleuder.net/munin/net/datenschleuder.net-if_dn42_rzl.html stehen Traffic-Statistiken des Uplinks.

Für interne Dienste steht die Blackbox von Inte zur Verfügung.

\section{Zentrale Benutzerverwaltung}

Die zentrale Benutzerverwaltung soll die Getränkekasse und die Tür-Codes verwalten.

\section{Türschloss}

Tim hat ein ABUS Hometech für 60€ bestellt.
Da der aktuelle Schließzylinder steht nicht weit genug über den Beschlag über, dieses Problem kann man auf zwei Arten lösen:
\begin{itemize}
\item{} Anderen Schließzylinder kaufen
\item{} Den Beschlag passen aussparen
\end{itemize}




Streaming von Vorträgen

\section{Benutzerdatenbank}

\section{Türschloss}

Welches Schloss?
Art der Authentifizierung?
Raumstatus?
Kommunikation der Mikrocontroller
Kassenterminal
Lichtsteuerung
Audioverdrahtung

Netzwerkstruktur
Streaming von Vorträgen

\end{document}
