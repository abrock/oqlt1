\documentclass{scrartcl}
\usepackage{xltxtra}
\begin{document}
\title{Protokoll der Jahreshauptversammlung des oqlt e.V.}
\date{20.06.2010}
\maketitle

\newcommand\MV{Mitgliederversammlung}

\tableofcontents
\hspace{1em}

15:38: Beginn der Mitgliederversammlung, Tobias Kral begrüßt die Anwesenden und verließt die Tagesordnung,
Alexander Brock wird als Protokollführer bestimmt. Tobias Kral fragt nach Anträgen und stellt die Beschlussfähigkeit fest,
es sind 8 von 14 Mitglieder anwesend.

\section{Bericht des Vorstandes:} {

Die letzte Mitgliederversammlung fand am 30.06.2009 statt, seitdem ist folgendes passiert:
\begin{itemize}
\item Einzug ins RZL
\item Gulasch-Programmier-Nacht 9 und 10
\item Easterhegg
\item Hacking at Random
\item Meta-Rhein-Main-Chaos-Days, von oqlt mitorganisiert.
\item SIGINT
\item 26C3: Chaos-Familien-Duell mit viel positiver Resonanz
\end{itemize}

Auch 2010 beteiligt sich oqlt an der Organisation der MRMCDS.
"oqlt am Sonntag" hat leider nicht mehr stattgefunden,
stattdessen aber ein "Tag des offenen Hackerspaces" mit 4 Vorträgen
Vereinsarbeit innerhalb von oqlt hat fast nicht stattgefunden,
da alle aktiven ihre gesamte verfügbare Freizeit darin investiert haben,
das RZL zu renovieren. Trotzdem wurde das Logo neu designed
und es wurde Kleidung mit dem neuen Logo angefertigt.

}
\section{Finanzbericht}
{
Das Wort geht an Tim Weber.

Tim Weber wurde als Kassenwart des oqlt e.V. gewählt, mit dem RZL ist der Arbeitsaufwand explodiert.
Die Buchung der eingereichten Quittungen ist wegen Zeitmangel 6 Monate im Rückstand.
Bei vielen Quittungen ist unbekannt, von wem sie eingereicht wurden,
bei vielen anderen, ob derjenige das Geld zurück oder eine Spendenquittung möchte.
Daher existiert zur Zeit kein vollständiger Kassenbericht, als grober Anhaltspunkt hier der Kassenstand:

\textbf{2235,79€}

Die Satzung schreibt keine Kassenprüfung vor, sie wird nur optional vorgesehen,
die \MV könnte Tim Weber daher als Schatzmeister entlasten.

Tim Weber stellt sich zur Wiederwahl, da er am ehesten noch den Überblick über
die offenen Quittungen hat.

Pro Woche kamen etwa 10 neue Quittungen, durch zeitliche Engpässe
kam es dazu, dass sich die Quittungen aufgestaut haben.

Regelmäßige Geldeingänge des RZL:
3 Leute spenden pro Monat insgesamt 90 €

Frage von Marcel Ackermann:
"Wie könnte man die Quittungsverwaltung vereinfachen"

Antwort:
"Sich mit den Leuten zusammensetzen und Quittungen durchgehen"

Anstehendes:
Übergabe der Finanzen an das RZL.

Vermögen, das dem RZL zusteht muss an den RZL e.V übertragen werden.
D.h. Geld auf dem Konto des oqlt e.V. abzüglich Mitgliedsbeiträge.

Darüber soll eine außerordentliche MV abstimmen.

Tim Weber schlägt vor, bei der außerordentlichen MV entlastet zu werden,
da zur Zeit kein vollständiger Kassenbericht vorliegt.
Dieser Vorschlag findet breite Zustimmung.

Die Finanzen des RZL sollen in naher Zukunft über ein Konto des RZL e.V.
laufen, frühestens aber wenn RZL e.V. eingetragener Verein und gemeinnützig ist.

Tobias Kral merkt an, dass oqlt die Satzungsgemäße Grenze der Ausgaben überschritten
wurde, indem Mate gekauft wurde.

Tim merkt an, dass diese Grenze nur im Innenverhältnis gilt,
wenn die MV keinen Zwergenaufstand probt stellt das aber kein Problem dar.

Weitere Fragen an den Vorstand? Keine.

}
\section{Entlastung des Vorstands}{

Das Wort geht an Tobias Kral.

Die Mitglieder des Vorstandes werden einzeln entlastet durch einfache Mehrheit mit Handzeichen

Niklas leitet die Entlastungswahl.
Entlastung des Kassenwarts: dafüer: 0, dagegen: 3, enthltung: 5
Entlastung des 1. Vorsitzenden (Tobias Kral) : 8 
Entlastung des 2. Vorsitzenden(Marcel Ackermann) dafür: 7, dagfegen: 0, enthaltung: 1

Tobias Kral dankt Niklas für seinen heroischen Einsaz bei der Entlastung des Vorstandes.

\section{Wahl des Vorstandes}

Vorschläge für den Kassenwart:
- Tim Weber

Vorschläge für die Vorsitzenden:
- Tobias Kral
- Sven Kreidermacher
- Alexander Brock
- Niklas Goerke

Niklas bereitet geheime Wahl vor.

Niklas möchte sich nicht zur Wahl stellen.

Jedes anwesende Mitglied hat zwei Stimmen für Vorsitzende und einen für den Schatzmeister.

Niklas und Timo zählen die Stimmen aus.

Tobias Kral: 8
Alex: 3
Tim:  8
Sven: 5

Tim nimmt die Wahl als Kassenwart des oqlt an.
Tobias Kral nimmt die Wahl zum 1. Vorsitzenden an.
Sven Kreidermacher nimmt die Wahl zum 2. Vorsitzenden an.

Das Wort geht an Tobias Kral.

Tobias Kral beantragt, dass oqlt e.V. sich offiziell am RZL beteiligen soll.

Tim will wissen, wozu das gut sein soll.

Tim: offene Aussprache.

Niklas ist dagegen, die Mitgliedsbeiträge von oqlt dem RZL zu übertragen,
da dadurch oqlt Handlungsunfähig würde.

Tim: Anteilige Übertragung der Mitgliedbeiträge?

Aktuelle Beiträge: 5€ + x, x€ R+

Tobias Kral: Abwägung zwischen Überleben des RZL und Vereinsleben des oqlt.

Dafo: 25% der Mitgliedsbeiträge.

Tim: oqlt hat fast kein Geld für Vereinsarbeit ausgegeben.

oqlt kann bei drohendem Tod des RZL immernoch Geld an das RZL übertragen.

Tobias: keine fixe Beiträge an das RZL, dadurch mehr Flexibilität.

Tim: Kontinuierliche Beteiliung sinnvoll.

Eine Umfrage ergibt 25% Beteiligung als Median der Stimmen.
Das wären bei 14 Mitgliedern 

Tim: Pro Monat macht das RZL 400-800€ Getränkeumsatz

Tobias Kral: Finanzieller Ausgleich als "Miete"

Die Mehrheit der Mitglieder ist für einen monatlichen Finanzausgleich von 25% der Mitgliedsbeiträge.

Tobias Kral fragt nach sonstigen Tagesordnungspunkten.

Niklas: orakel.
Marcel: Zukünftige Planung.

Tobias Kral:

Die MV gestattet dem Vorstand mit einfacher Mehrheit rückwirkend den Ankauf von Getränken in beliebiger Menge
auch ...

}
\section{Sonstiges}{

6.1 Orakel.

Tobias Kral:
Früher: git-repos auf dem orakel, seit langer Zeit offline.

Problem: orakel braucht viel Strom, stimmt für 

Niklas: Prinzipiell: Serverhousing ja/nein?

Tobias Kral: Website von oqlt wird nicht gepflegt sei 1/2y

docsteel bietet an, eine VM für oqlt auf einem seiner Rootserver anzubieten.
- fett angebunden
- 4TB
- IPv6

Abstimmung: 7 dafür, 1 enthaltung

6.2 Zukünftige Planung

Kalender?

Tim: Angefangene Kalendersoftware vorhanden.

Usecases unklar.
Dringende Angelegenheit.

Dafo: Will Kalender per Brief.

Tim: 3€ pro Monat.

CFD:
Software veröffentlichen.

Vorbereitung für nächsten Kongress.

Marcel:
oqlt am Sonntag wiederbeleben.
inhaltliche Arbeit wiederaufnehmen.

Sven Koodiniert Themensammlung für oqlt am Sonntag.

Tobias Kral: Neuer Flyer notwendig.

Umzug der Domain von Tobias Kral nach oqlt e.V.

Sitzung geschlossen, 17:10

}


\end{document}